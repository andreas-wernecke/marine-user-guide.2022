\section{Data model and formats} \label{data_model}
Data from the Climate Data Store are returned in delimited text files (.csv), with 19 columns per row and one observed value per row. 
The columns are listed in table \ref{tab:cdm_lite_main} together with a brief description.  
The date and time of the observation is given by the \code{date\_time} column, with the value returned as a string of the format \code{YYYY-MM-DD hh:mm:ss}. 
All data are returned in the UTC time zone.
The location of the observation is given by the \code{latitude} and \code{longitude} columns in the WGS84 / EPSG4326 coordinate reference system. 
The variable being reported / observed is given by the \code{observed\_variable} column, the units by the \code{units} column and the observed value by the \code{observation\_value} column. 
The remaining columns provide further contextual information for the observations, such as platform type and station identifiers etc. 
% A number of the columns are enumerated using code tables, indicated by the text \code{(coded)} in the Kind column of Table \ref{tab:cdm_lite_main}.
% These code tables are reproduced in tables \ref{tab:report_type_main} to \ref{tab:licence_main} but with only those values used in the marine data listed.

\csvreader[
  longtable=|p{0.25\linewidth}|p{0.2\linewidth}|p{0.45\linewidth}|,
  table head=\caption{Data model.\label{tab:cdm_lite_main}}\\
    \hline\bfseries Field &\bfseries Kind &\bfseries Description \\ \hline\endfirsthead
    \multicolumn{3}{c} {{\tablename\ \thetable{} -- continued from previous page}} \\ 
    \hline\bfseries Field &\bfseries Kind &\bfseries Description \\ \hline\endhead
    \hline\endfoot,    
    separator=semicolon,
    late after line=\\ \hline
]{./data_model/cdm_lite.csv}{1=\Field,2=\Kind,3=\Description}{\Field & \Kind & \Description }
